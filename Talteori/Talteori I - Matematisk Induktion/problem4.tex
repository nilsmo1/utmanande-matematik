\subsection*{Problem 4}
\addcontentsline{toc}{subsection}{\protect\numberline{}Problem 4}

\paragraph{Problem:} Visa att för alla $n\ge 0$ är \(\sum_{i=0}^nF_i^2=F_nF_{n+1}\).

\paragraph{Lösning:}\\
$F_n$ är det n:te fibonaccitalet. Fibonaccitalen definieras som:
\begin{equation*}
\hspace{-75pt}
F_n=
    \begin{cases}
        0 & n=0\\
        1 &  n=1\\
        F_{n-1}+F_{n-2} & n> 1
    \end{cases}
\end{equation*}

\paragraph{Induktionsbas (IB)} $n=0$:
\begin{align*}
VL &= \sum_{i=0}^0F_i^2 = F_0^2 = 0^2 = 0 \\[5pt]
HL &= F_{0}F_{0+1} = F_0F_1 = 0\cdot1 = 0 \\[10pt]
VL &= 0 = HL
\end{align*}

\paragraph{Induktionsantagande (IA)} Antag att \(\sum_{i=0}^kF_i^2=F_kF_{k+1}\) för $k \ge0$.

\paragraph{Induktionssteg (IS)} Visa att \(\sum_{i=0}^{k+1}F_i^2=F_{k+1}F_{k+2}\).\newline
Vi börjar med att dela upp summan i VL enligt:
\[\sum_{i=0}^{k+1}F_i^2 = (\sum_{i=0}^{k}F_i^2) + F_{k+1}^2\]
Enligt (IA) gäller \(\sum_{i=0}^{k}F_i^2 = F_kF_{k+1}\) och vi kan därför göra följande omskrivning:
\[(\sum_{i=0}^{k}F_i^2) + F_{k+1}^2 = (F_kF_{k+1})+F_{k+1}^2\]
Om vi nu bryter ut $F_{k+1}$ ur båda termer får vi:
\[F_{k+1}(F_k+F_{k+1})\]
Enligt definitionen av fibonaccital från problembeskrivningen är summan av två på varandra följande fibbonacital det nästa talet i fibbonaciserien. Notera därför att $F_k+F_{k+1} = F_{k+2}$. Uttrycket kan därför skrivas om enligt:

\[F_{k+1}(F_k+F_{k+1}) = F_{k+1}F_{k+2} \]
\\[10pt]
\[\sum_{i=0}^{k+1}F_i^2 = F_{k+1}F_{k+2}\]
\hfill\qed