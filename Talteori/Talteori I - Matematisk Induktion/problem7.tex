\subsection*{Problem 7}
\addcontentsline{toc}{subsection}{\protect\numberline{}Problem 7}

\paragraph{Problem:} En talföljd $a_0,\ a_1,\ a_2,\ \dotsc$ definieras för alla heltal $n\ge 0$ genom $a_0=0$ och därefter \(a_{n+1}=\frac 1{2-a_n}\), för $n\ge 0$. Gissa en explicit formel för $a_n$ och bevisa den sedan med induktion.

\paragraph{Lösning:}

\paragraph{Arbete mot gissning} Vi börjar med att räkna ut de första 4 talen med hjälp av att vi vet att $a_0 =0$:
\begin{align*}
a_1=\frac{1}{2-a_0}&=\frac{1}{2-0}=\frac{1}{2} \\[6pt]
a_2=\frac{1}{2-a_1}&=\frac{1}{2-\frac{1}{2}}=\frac{1}{\frac{3}{2}}=\frac{2}{3} \\[6pt]
a_3=\frac{1}{2-a_2}&=\frac{1}{2-\frac{2}{3}}=\frac{1}{\frac{4}{3}}=\frac{3}{4} \\[6pt]
a_4=\frac{1}{2-a_3}&=\frac{1}{2-\frac{3}{4}}=\frac{1}{\frac{5}{4}}=\frac{4}{5} \\
\end{align*}
Från detta kan vi notera ett mönster och göra vår gissning för den explicita formeln:
\[a_n=\frac{n}{n+1}\]
\paragraph{Induktionsbas (IB)} $n=0$:
\[a_{0}=\frac{0}{0+1}=\frac{0}{1}=0\]\\
Detta stämmer överens med faktumet att $a_0=0$ från problembeskrivningen.

\paragraph{Induktionsantagande (IA)} Antag att $a_k=\frac{k}{k+1}$ för $k\ge0$.

\paragraph{Induktionssteg (IS)} Visa att $a_{k+1}=\frac{k+1}{k+2}$.
Vi börjar med att kombinera den gissade formeln med den rekursivt definierade formeln från problembeskrivningen. Vi sätter sedan in värdet på $a_k$ taget från (IA):

\[a_{k+1} &= \frac{1}{2-a_k} &&= \frac{1}{2-\frac{k}{k+1}} = \frac{1}{\frac{2(k+1)-k}{k+1}} = \frac{1}{\frac{k+2}{k+1}} = \frac{k+1}{k+2}\]
\\[10pt]
\[a_{k+1}=\frac{k+1}{k+2}\]
\hfill\qed

