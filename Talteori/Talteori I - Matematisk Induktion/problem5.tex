\subsection*{Problem 5}
\addcontentsline{toc}{subsection}{\protect\numberline{}Problem 5}

\paragraph{Problem:} Talföljden $a_0,\ a_1,\ a_2,\ \dotsc$ definieras rekursivt genom: \newline \(a_0=7,\ a_1 =5\) och \(a_{n+1} = 2a_n + 3a_{n-1}\) för $n\ge 1$. Du misstänker att det finns två konstanter $A$ och $B$ sådana att $a_n =A\cdot 3^n + B\cdot (-1)^n$ för alla $n\ge 0$. Verifiera misstanken (finn möjliga $A$ och $B$) och bevisa ditt påstående med induktion.

\paragraph{Lösning:}

\paragraph{Beräkning av A och B:} Vi sätter in $n=0$ och $n=1$ i den explicita formeln för $a_n$ och ställer upp ett ekvationssystem med hjälp av givna värden på $a_0$ och $a_1$:
\begin{alignat*}{4}
    a_0 &= 7 &&= A\cdot3^0+B\cdot(-1)^0 = &&&A+B \\[5pt]
    a_1 &= 5 &&= A\cdot3^1+B\cdot(-1)^1 = 3&&&A-B \\
\end{alignat*}
Detta ger ekvationssystemet:\\
\[\systeme{A+B = 7, 3A-B = 5}\]
Vi använder additionsmetoden för ekvationssystem för att beräkna variablerna:
\begin{gather*}
A+3A+B-B = 7 + 5 \\[5pt]
4A = 12 \\[5pt]
A = 3
\end{gather*}
Om vi sätter in det beräknade värdet på $A$ i ekvationssystemet får vi:
\begin{alignat*}{3}
&A+B &&= 7 \\[5pt]
&3+B &&= 7 \\[5pt]
&B &&= 4
\end{alignat*}
Vi har nu fått fram värden på $A$ och $B$. Vi kan därmed konstatera att den explicita formeln för $a_n$ är:
\[a_n = 3\cdot 3^n + 4\cdot(-1)^n = 3^{n+1}+4(-1)^n\]
Vi ska nu visa att denna formel gäller för $n\ge 0$ med induktion.

\paragraph{Induktionsbas (IB)} $n=0$ och $n=1$:
\begin{align*}
a_0=3^{0+1}+4(-1)^0 = 3+4 = 7 \\[5pt]
a_1=3^{1+1}+4(-1)^1 = 9-4 = 5
\end{align*}
Fallen stämmer enligt problembeskrivningen och vi går vidare.

\paragraph{Induktionsantagande (IA)} Antag att $a_k=3^{k+1}+4(-1)^k$ för $k\ge 0$.

\paragraph{Induktionssteg (IS)} Visa att $a_{k+1}=3^{k+2}+4(-1)^{k+1}$.\\
Vi använder oss av den rekursiva formeln given i problembeskrivningen och (IA) för att göra induktionssteget:
\begin{alignat*}{3}
a_{k+1} &= 2a_k+3a_{k-1} \\[5pt]
&= 2(3^{k+1\hspace{4pt}}+4(-1)^k) + 3(3^{k}+4(-1)^{k-1})\\[5pt]
&= 2\cdot3^{k+1}+3\cdot 3^k+8(-1)^k\hspace{1pt}+12(-1)^{k-1} \\[5pt]
&= 2\cdot 3^{k+1}+3^{k+1}\hspace{3pt}+ (8-12)(-1)^{k} \\[5pt]
&= 3\cdot 3^{k+1}-4(-1)^{k} = 3^{k+1} + 4(-1)(-1)^k \\[5pt]
&= 3^{k+2}\hspace{15pt}+4(-1)^{k+1} \\[35pt]
a_{k+1} &= 3^{k+2}+4(-1)^{k+1}
\end{alignat*}
\hfill\qed


