\subsection*{Problem 8}
\addcontentsline{toc}{subsection}{\protect\numberline{}Problem 8}

\paragraph{Problem:} Använd induktion för att visa att för alla $n\ge 1$ är talet $2^{2n}-1$ delbart med 3.

\paragraph{Lösning:}

\paragraph{Induktionsbas (IB)} $n=1$:
\[2^{2\cdot1}-1 = 2^2-1 = 4-1 = 3\]
vilket givetvis är delbart med 3.

\paragraph{Induktionsantagande (IA)} Antag att $2^{2k}-1$ är delbart med 3, dvs. $2^{2k}-1 = 3q$ där $q \in \mathbb{N}$ för $k\ge1.$ 

\paragraph{Induktionssteg (IS)} Visa att $2^{2(k+1)}-1$ är delbart med 3.\newline
Uttrycket kan expanderas till
\[2^{2k+2}-1 = 2^2\cdot2^{2k} = 4\cdot2^{k}-1\]
Vi kan sedan gruppera termerna enligt:
\[4\cdot2^{2k}-1 = 3\cdot2^{2k} + 2^{2k}-1\]
Enligt (IA) är $2^{2k}-1$ delbart med 3 och kan skrivas om till $3q$. Därför kan uttrycket skrivas om till:
\[3\cdot2^{2k} + 2^{2k}-1 = 3\cdot2^{2k} + 3q = 3(2^{2k}+q)\]
\[2^{2(k+1)}-1=\boldsymbol{3}(2^{2k}+q) \equiv 0 \pmod{3}\]
\hfill\qed